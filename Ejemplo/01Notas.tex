\documentclass[a4,fnuevo]{aleph-notas}

% -- Paquetes adicionales
\usepackage{enumitem}
\usepackage{aleph-comandos}

% -- Datos del libro
\autor{Andrés Merino}
\materia{Cálculo Vectorial}
\nota{Resumen no. 1}
\tema{Cambio de coordenadas}
\fecha{Agosto de 2020}

\logodos[1.0cm]{Logos/logo01}
\logouno[2.9cm]{Logos/logo02}



\begin{document}

\encabezado

\begin{defi}[Coordenadas polares]
    En $\mathbb{R}^2$, el cambio a coordenadas polares es la función
    \[
        \funcion{P}{\mathbb{R}^2}{\mathbb{R}^2}
        {(r,\theta)}{P(r , \theta)=(x,y),}
    \]
    donde
    \[
        x= r\cos(\theta)
        \texty
        y= r\sen(\theta)
    \]
    para $(r,\theta)\in \mathbb{R}^2$.
\end{defi}

\begin{advertencia}
    Notemos que si $(x,y)=P(r,\theta)$, entonces se tiene que
    \[
        r^2=x^2+y^2
        \texty
        \tan(\theta)=\frac y x,
    \]  
    para todo $(r,\theta)\in\mathbb{R}^2$ tal que $x\neq 0$.
\end{advertencia}


\begin{ejer}
    Calcular \[\iint_D x^2+y^2\,dxdy\] donde $D=\{(x,y\in\mathbb{R}^2:x^2+y^2\leq 4,\ y>0\}$.
\end{ejer}

\begin{proof}[Solución]
    Para resolver esta integral, tomaremos el cambio de variable a coordenadas polares, es decir, el cambio de variable
    \[
        x=r\cos(\theta)
        \texty
        y=r\sen(\theta),
    \]
    es decir, tendremos la función definida por
    \[
        P(r,\theta) = (r\cos(\theta) , r\sen(\theta)),
    \]
    para $(r,\theta)\in\mathbb{R}^2$, con lo cual
    \[
        \det(J_P(r,\theta)) = r
    \]
    para $(r,\theta)\in\mathbb{R}^2$. Ahora, necesitamos encontrar $D^*$ tal que $P(D^*)=D$. Dado que $D$ es el semicírculo de centro en el origen y radio 2, podemos ver que
    \[
        D^*=[0,2]\times [0,\pi].
    \]
    Así, tenemos que
\end{proof}


\end{document}